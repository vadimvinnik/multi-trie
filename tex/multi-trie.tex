%-------------------------------------------------------------------
%
% Author    : Vadim Vinnik
% E-mail    : vadim.vinnik@gmail.com
% Date      : 2015-12-03
% Status    : Draft
% License   : Creative commons
% Keywords  : name, value, map, trie, multi-set, haskell, monad
%
% Summary
% A ``trie of multisets'' is defined as a mathematical concept that
% reflects some aspects of structured data objects in programming;
% Properties thereof are investigated; An implementation in Haskell
% is described.
%
%-------------------------------------------------------------------

\documentclass{article}

\usepackage{amsthm}
\usepackage{amsfonts}

\newtheorem{Df}{Definition}
\newtheorem{Th}{Theorem}
\newtheorem{Pf}{Proof}
\newtheorem{Rm}{Remark}

\title{Tries of multisets, their properties, applications and implementation}
\author{Vadim Vinnik}
\date{2015}

\begin{document}

\maketitle

\begin{abstract}
A ``trie of multisets'' is defined as a mathematical concept that
reflects some aspects of structured data objects in programming;
Properties thereof are investigated; An implementation in Haskell
is described.
\end{abstract}

\tableofcontents

\section{Introduction}

Suppose there is a given set~$D$ whose elements are called \emph{values}, or
\emph{denotata}, and a set~$V$ of objects called \emph{names}.

A quite natural formalisation of \emph{naming} as a relation between names and
their values is the following
\begin{Df}
A \emph{naming set} is a partial mapping $s: V\to D$.
\end{Df}

In other words, a naming set is an object of the form
\[
  s = \{ (v_1, d_1), (v_2, d_2), \ldots, \} ,
\]
where $v_i\in V$, $d_i\in D$ and all names~$v_i$ are pairwise distinct. The
latter requirement formalizes \emph{unambiguity}, or \emph{univaluedness}: a
name cannot have different values in a given context.

Naming mappings defined as above are widely used in theoretical computer
science for syntactical as well as semantical tasks. Maps as direct
implementations of naming sets are included into standard libraries of various
languages and platforms and widely used in practice.

At the topmost abstraction level, a naming set represents a plain naming
relation where names and values are atomic in the sense that their internal
structures and any non-trivial properties are hidden as being irrelevant. In
fact, the only special property taken into account by the definition above is
equatability of names: for any two names it should be possible to decide
whether they are identical.

Introducing various properties of names and/or denotata, one can obtain a
number of interesting and useful specialised formalisations of naming for
theoretical purposes as well as data structures implementing them.
This article describes a case where
\begin{itemize}
\item names form a monoid, i.e. compound names can be concatenated from
simpler ones;
\item values are sets (replaced by sequences in the implementation).
\end{itemize}

\section{Definition and basic properties}

If~$X$ is a set, let~$X^\ast$ denote a set of all sequences of its elements,
and~$2^X$ be a set of all its subsets.

Let~$V$ be a given set of objects called \emph{atomic names}. Elements
of~$V^\ast$ are called \emph{compound names}. Let us omit the words ``atomic''
and ``compound'' if it does not lead to confusion.

\begin{Df}
An \emph{$(V,D)$-multitrie} is a mapping
\[
  p : V^\ast \to 2^D .
\]
\end{Df}

In other words, a multi-trie maps complex names, i.e. sequences of
atomic names, into sets of values.

Whenever $N$ and $V$ are obvious from the context, we'll omit ``$(N,V)$-''
prefix and write simply ``multitrie''.

\section{Notes about implementation}

\section{Application}

\end{document}
