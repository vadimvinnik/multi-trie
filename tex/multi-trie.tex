%-------------------------------------------------------------------
%
% Author    : Vadim Vinnik
% E-mail    : vadim.vinnik@gmail.com
% Date      : 2015-12-03
% Status    : Draft
% License   : Creative commons
% Keywords  : name, value, map, trie, multi-set, haskell, monad
%
% Summary
% A ``trie of multisets'' is defined as a mathematical concept that
% reflects some aspects of structured data objects in programming;
% Properties thereof are investigated; An implementation in Haskell
% is described.
%
%-------------------------------------------------------------------

\documentclass{article}

\usepackage{amsfonts}
\usepackage{amssymb}
\usepackage{amsthm}

\theoremstyle{definition}
\newtheorem{Df}{Definition}
\newtheorem{Th}{Theorem}
\newtheorem{St}{Statement}
\newtheorem{Pf}{Proof}
\newtheorem{Rm}{Remark}

\newcommand{\setcharmvcn}{M}
\newcommand{\setcharsvcn}{S}
\newcommand{\setcharmt}{T}

\newcommand{\setsymbol}[3]{\mathcal{#1}_{#2,#3}}

\newcommand{\setmvcn}[2]{\setsymbol{\setcharmvcn}{#1}{#2}}
\newcommand{\setsvcn}[2]{\setsymbol{\setcharsvcn}{#1}{#2}}
\newcommand{\setmt}[2]{\setsymbol{\setcharmt}{#1}{#2}}

\newcommand{\flatten}{\mathrm{Fl}}
\newcommand{\select}{\mathrm{Sel}}
\newcommand{\singleleaf}{\mathrm{Sg}}

\newcommand{\deref}[2]{\mathrm{Get}(#1, #2)}

\title{Tries of multisets, their properties, applications and implementation}
\author{Vadim Vinnik}
\date{2015}

\begin{document}

\maketitle

\begin{abstract}
A ``trie of multisets'' is defined as a mathematical concept that
reflects some aspects of structured data objects in programming;
Properties thereof are investigated; An implementation in Haskell
is described.
\end{abstract}

\tableofcontents

\section{Introduction}

%todo:
Why naming is so important. Why its formalisation is needed.

Suppose there is a given set~$D$ whose elements are called \emph{values}, or
\emph{denotata}, and a set~$V$ of objects called \emph{names}.

A quite natural formalisation of \emph{naming} as a relation between names and
their values is the following
\begin{Df}\label{df:naming-set}
A \emph{naming set} is a partial mapping $s: V\to D$.
\end{Df}

In other words, a naming set is an object of the form
\[
  s = \{ (v_1, d_1), (v_2, d_2), \ldots, \} ,
\]
where $v_i\in V$, $d_i\in D$ and all~$v_i$ are pairwise distinct. The
latter requirement formalizes \emph{unambiguity}, or \emph{univaluedness}: a
name cannot have different values in a given context.

Naming mappings defined as above are widely used in theoretical computer
science for syntactical as well as semantical tasks.
%todo: references
Maps as direct
implementations of naming sets are included into standard libraries of various
programming languages and platforms and widely used in practice.
%todo: references

At the topmost abstraction level, a naming set represents a ``plain''
relation where names and values are atomic in the sense that their internal
structures and any non-trivial properties are hidden as irrelevant. In
fact, the only special property taken into account by the definition above is
equatability of names: for any two names it should be possible to decide
whether they are identical.

Introducing various properties of names and/or denotata, one can obtain a
number of interesting and useful specialised formalisations of naming for
theoretical purposes as well as implementable data structures suitable for
practical tasks. This article describes a kind of naming with
two differences from the definition~\ref{df:naming-set}:
\begin{itemize}
\item multivaluedness: any name can be related to zero or more values;
\item compoundness of names: names form a monoid i.e. compound names could
be concatenated from simpler ones.
\end{itemize}


\section{Abstract definition and basic properties}

If~$X$ is a set, let~$X^\ast$ denote a set of all sequences of its elements
(also known as \emph{chains}), and~$2^X$ be a set of all its subsets.
If~$a,b\in X^\ast$ are two chains, their concatenation will be denoted simply
as~$ab$. Let~$\varepsilon$ stand for the empty chain.

Let~$V$ be a given set of objects called \emph{atomic names}. Elements
of~$V^\ast$ are called \emph{compound names}. Let us omit the words ``atomic''
and ``compound'' if it does not lead to confusion.

\begin{Df}\label{df:mvcn}
A \emph{$(V,D)$-multi-valued compound-naming set (MVCNset)} is a binary
relation
\[
  p \subseteq V^\ast \times D .
\]
Whenever $V$ and $D$ are obvious from the context, we'll omit ``$(V,D)$-''
prefix and write simply ``MVCNset''. A set of $(V,D)$-MVCNsets will be
denoted~$\setmvcn{V}{D}$.
\end{Df}

Dereferencing distributes over set-theoretical operations:
\begin{St}\label{st:mvcn-deref-distributivity}
Let~$\odot$ stand for either~$\cup$ or~$\cap$, and let~$p$ and~$q$ be
$(V,D)$-MVCNsets. Then
\[
  \deref{p\odot q}{v} = \deref{p}{v} \odot \deref{q}{v} ,
\]
for every~$v\in V^\ast$.
\end{St}

Obviously, $\setmvcn{V}{D}$ is closed against set-theoretical union
and intersection:
\begin{St}\label{st:mvcn-setop}
If~$p$ and~$q$ are $(V,D)$-MVCNsets, so are~$p\cup q$ and~$p\cap q$.
\end{St}

For (univalued) naming sets, to dereference a name means to find its only
value, if any.  In a similar way, the following operation finds all values
of a name in an MVCNset, no matter how many.

\begin{Df}\label{df:mvcn-dereferencing}
Given a $(V,D)$-MVCNset~$p$ and a name~$v\in V^\ast$, \emph{dereferencing}
of~$v$ in~$p$ is
\[
  \deref{p}{v} = \{ d | (v, d) \in p \} .
\]
\end{Df}

Let us introduce special terms and symbols for the two extreme cases of
MVCNsets, namely:
\begin{Df}\label{df:mvcn-extremes}
MVCNsets $\bot$ called \emph{empty} and $\top$ called \emph{full} are defined by
\begin{eqnarray*}
  &  \bot = \varnothing ; \\
  &  \top = V^\ast \times D .
\end{eqnarray*}
\end{Df}

Obviously,
\begin{St}\label{st:mvcn-extreme-deref}
For any~$v\in V^\ast$,
\begin{eqnarray*}
  & \deref{\bot}{v} = \varnothing, \\
  & \deref{\top}{v} = D .
\end{eqnarray*}
\end{St}

It is easy to see that~$\bot$ and~$\top$ are neutral elements
of~$\cup$ and~$\cap$, respectively:
\begin{St}\label{st:mvcn-neutrals}
For any MVCNset~$p$,
\begin{eqnarray*}
  & \bot \cup p = p, \\
  & \top \cap p = p .
\end{eqnarray*}
\end{St}

Note that a compound name~$v$ in a MVCNset~$p$ not only is a reference to
multiple values but also is a common prefix for a ``bunch'' of names starting
with~$v$. In this sence,~$v$ is a name for an entire selection MVCNset that is
formalised in the following
\begin{Df}\label{df:mvcn-select}
Let~$p\in\setmvcn{V}{D}$, $v\in V^\ast$, then \emph{selection} from~$p$
under~$v$ is
\[
  \select(p,v) = \{ (w, d) \mid (vw, d)\in p \} .
\]
\end{Df}

The followig properties of selection are obvious:
\begin{St}\label{st:mvcn-selection-properties}
For any~$p,q\in\setmvcn{V}{D}$, $u, v\in V^\ast$, $\odot\in\{\cup, \cap\}$,
\begin{eqnarray*}
  & \select(p,\varepsilon) = p, \\
  & \select(p,uv) = \select(\select(p,u), v), \\
  & \select(p\odot q, v) = \select(p,v)\odot \select(q,v).
\end{eqnarray*}
\end{St}

It is interesting to note that there is another formalization of the
multivalued naming that is, however, equivalent to the above definitions.  It
is described in the next section.

\section{Alternative definition}

The formalization of multivalued naming with compound names presented in the
previous section was straightforward in the sense that the definition directly
reflected the notion. However, the same substantial object could be modelled
through a naming set (in the sense of definition~\ref{df:naming-set}, i.e.
uniuvalued) whose values are subsets of~$D$.

\begin{Df}\label{df:svcn}
A \emph{$(V,D)$-set-valued compound-naming set ($(V,D)$-SVCNset)} is a total
mapping
\[
  p : V^\ast \to 2^D .
\]
A set of $(V,D)$-SVCNsets will be denoted~$\setsvcn{V}{D}$.\qed
\end{Df}

The only special detail here is totality of~$p$: every compound name has some
value~-- maybe an empty set.

Dereferencing needed a special definition (def.~\ref{df:mvcn-dereferencing})
for MVCNsets. With SVCNsets, however, it is just an alias for a functional
application of an SVCNset, as a functional binary relation, to a name:
\begin{Df}\label{df:svcn-dereferencing}
Given a $(V,D)$-SVCNset~$p$ and a name~$v\in V^\ast$, \emph{dereferencing}
of~$v$ in~$p$ is
\[
  \deref{p}{v} = p(v).
\]
\end{Df}

Union, intersection operations as well as empty and full namings whose
definitions were straightforward for MVCNsets (see st.~\ref{st:mvcn-setop}
and def.~\ref{df:mvcn-extremes}) need a more elaborate definitions in the
SVCNset world.

\begin{Df}\label{df:union-intersection}
Let~$\odot$ stand for any of~$\cup$ and~$\cap$, and let~$p$ and~$q$ be
$(V,D)$-SVCNsets. Then their \emph{union} and \emph{intersection} are
$(V,D)$-SVCNsets, such that
\[
  p\odot q = \{ (v, p(v) \odot q(v)) \mid v\in V^\ast \} .
\]
\end{Df}

\begin{Df}\label{df:svcn-extremes}
SVCNset~$\bot$ called \emph{empty} and $\top$ called \emph{full} are defined by
\begin{eqnarray*}
  &  \bot = \{ (v, \varnothing) \mid v\in V^\ast \} ; \\
  &  \top = \{ (v, D) \mid v\in V^\ast \} .
\end{eqnarray*}
\end{Df}

Properties similar to those of MVCNsets from st.~\ref{st:mvcn-extreme-deref}
are, for SVCNSets, just a rewritten definition of~$\bot$ and~$\top$.
Properties similar to st.~\ref{st:mvcn-neutrals} hold for SVCNsets.

%todo: stopped here

\begin{Df}
Let~$v$ be some compound name and~$A$ be some set of values, $v\in V^\ast$,
$A\subseteq D$.  Then~$p = \singleleaf(v,A)$ is such an SVCNset that~$p(v) =
A$ and~$p(u) = \varnothing$ for any $u\in V^\ast$, $u\neq v$.
\end{Df}

\begin{Df}\label{df:select}
Let~$p\in\setmt{V}{D}$, $v\in V^\ast$, then
\[
  \select(p,v) = \{ (w, p(vw)) \mid w\in V^\ast \} .\qedhere
\]
\end{Df}

Obviously, for any multi-trie~$p$ and compound names~$u,v$:
\begin{eqnarray*}
  & \select(p,\varepsilon) = p ,\\
  & \select(p,uv) = \select(\select(p,u), v) .
\end{eqnarray*}

Obviously, empty and full multitries are neutral elements of union and intersection,
respectively. These operations are commutative, associative and idempotent.
Moreover, these operations are distributive against selection:
\[
  \select(p\odot q, v) = \select(p, v) \odot \select(q, v) .
\]

\begin{Df}\label{df:cartesian}
Let~$p$ and~$q$ be $(V,D')$- and $(V,D'')$-multitries, respectively. Their
\emph{cartesian product} $p\times q = r$ is a $(V,D'\times D'')$-multitrie
such that
\[
  r(w) = \bigcup_{u,v\in V^\ast: uv = w} p(u) \times q(v)
\]
for any~$w\in V^\ast$.\qed
\end{Df}

Obviously,~$\times$ is distributive against $\cap$ and $\cup$:
\[
  p\times(q\dot r) = p\times(q\dot r)
\]

Consider a multitrie whose values are, in their turn, multitries.
Flattening operation formally defined below turns it into a ``plain'' multitrie.
\begin{Df}\label{df:flatten}
\emph{Flattening} is an unary operation $\flatten : \setmt{V}{\setmt{V}{D}}
\to\setmt{V}{D}$, such that, for any $(V,\setmt{V}{D})$-multitrie~$p$, the
corresponding $r=\flatten(p)$ is defined by
\[
  r(w) = \bigcup_{u,v\in V^\ast: uv = w} p(u)(v)
\]
for any name~$w\in V^\ast$.\qed
\end{Df}

It is easy to see that the above definitions are equivalent to the following:
\begin{eqnarray}
  \label{eq:alt-cartesian}
  p\times q =
    \bigcup_{u,v\in V^\ast} \singleleaf(uv, p(u) \times q(v)) ,\\
  \label{eq:alt-flatten}
  \flatten(p) =
    \bigcup_{u,v\in V^\ast} \singleleaf(uv, p(u)(v)) .
\end{eqnarray}
(note that~$\cup$ in the definitions~\ref{df:cartesian} and~\ref{df:flatten}
means a set-theoretical union whereas in~(\ref{eq:alt-cartesian})
and~(\ref{eq:alt-flatten}) it means union of multitries, see
def.~\ref{df:union-intersection}).

\section{Constructional approach}

The above definition is abstract in the sense that it only describes how do
multitries look like from outside~-- namely, as a correspondence between
compound names and their values.

\section{Notes about implementation}

\section{Application}

\end{document}

