%-------------------------------------------------------------------
%
% Author    : Vadim Vinnik
% E-mail    : vadim.vinnik@gmail.com
% Date      : 2015-12-03
% Status    : Draft
% License   : Creative commons
% Keywords  : name, value, map, trie, multi-set, haskell, monad
%
% Summary
% A ``trie of multisets'' is defined as a mathematical concept that
% reflects some aspects of structured data objects in programming;
% Properties thereof are investigated; An implementation in Haskell
% is described.
%
%-------------------------------------------------------------------

\documentclass{article}

\usepackage{amsthm}
\usepackage{amsfonts}

\newtheorem{Df}{Definition}
\newtheorem{Th}{Theorem}
\newtheorem{Pf}{Proof}
\newtheorem{Rm}{Remark}

\title{Tries of multisets, their properties, applications and implementation}
\author{Vadim Vinnik}
\date{2015}

\begin{document}

\maketitle

\begin{abstract}
A ``trie of multisets'' is defined as a mathematical concept that
reflects some aspects of structured data objects in programming;
Properties thereof are investigated; An implementation in Haskell
is described.
\end{abstract}

\tableofcontents

\section{Introduction}

\section{Definition and basic properties}

\begin{Df}
Let~$N$ be a set of objects called \emph{names}, $V$~be a set of objects called
\emph{values}. Then an \emph{$(N,V)$-multitrie} is a mapping
\[
  p : N^* \to V^*
\]
\end{Df}

\begin{Rm}
In other words, a multi-trie maps complex names, i.e. sequences of
atomic names from~$N$, into sequences of values.
\end{Rm}

\begin{Rm}
Whenever $N$ and $V$ are obvious from the context, we'll omit ``$(N,V)$-''
prefix and write simply ``multitrie''.
\end{Rm}

\begin{Rm}
At the current high abstraction level, the only property required from~$N$
is an ability to decide whether two arbitrary names are the same or
different~-- that is an obvious implicit assumption. Linear ordering of
names is important for an efficient implementation but does not add anything
to the mathematical properties of the objects and, therefore, is now
ignored.
\end{Rm}

\begin{Rm}
Indeed, multitries arising from any practical computation would be finite.
However, we do not impose this restriction for now~-- all considerations
below hold for finite as well as infinite multitries.
\end{Rm}

This definition presents an abstract, topmost view of the objects being
described. Another view based on a natural implementation is discussed later,
in section~\ref{sec:constructive}.

\section{Constructive approach}\label{sec:constructive}

\section{Link to mathematical models of naming}

\section{Notes about implementation}

\section{Application}

\end{document}
